\documentclass[12pt]{article}
\usepackage[utf8]{inputenc}
\usepackage[T1]{fontenc}
\usepackage{indentfirst}
\usepackage{amsmath}
\usepackage{amssymb}
\usepackage{natbib}
\usepackage{graphicx}
\usepackage{float}
\usepackage[a4paper, margin = 2 cm]{geometry}
\usepackage{fancyhdr}
\usepackage{wrapfig}
\usepackage{hyperref}
\usepackage{mathtools}

\title{Computational complexity assignment 2}
\author{Dominik Wawszczak}
\date{2024-10-31}

\begin{document}
	\setlength{\parindent}{0 cm}
	
	Dominik Wawszczak \hfill Computational Complexity
	
	student id number: 440014 \hfill assignment 2
	
	group number: 3
	
	\bigskip
	\hrule
	\bigskip
	
	\textbf{Task 1}
	
	\medskip
	
	To begin, we will show that verifying whether an input contains a tree is
	indeed in \(\mathbf{L}\). A graph is a tree if and only if it meets the
	following two conditions:
	\begin{enumerate}
		\item The number of edges is one less than the number of vertices.
		\item The graph is connected.
	\end{enumerate}
	Checking the first condition is straightforward. We can iterate over the
	edges while maintaining a binary counter and then verify if it equals
	\(n - 1\). This counter will require no more than \(\mathcal{O}(N)\)
	space, where \(N\) is the size of the input.
	
	\medskip
	
	The second condition is more complex. We can iterate over all pairs \((s, t)
	\in \{1, \ldots, n\}^{2}\) using two binary counters. For each pair, we need
	to determine if a simple path exists from \(s\) to \(t\). This problem is in
	\(\mathbf{L}\), as discussed in the lecture.
	
	\medskip
	
	A \textit{root} of a tree with radius at most \(2\) is a vertex \(r \in
	V(T)\) such that every other vertex in \(T\) is within distance \(2\) of
	\(r\).
	
	\medskip
	
	We will use a slightly different definition of an \textit{induced subgraph}
	than the one in the problem statement. A graph \(G_{1}\) is an induced
	subgraph of \(G_{2}\) if and only if there exists an injection \(\mu :
	V(G_{1}) \to V(G_{2})\) such that
	\[ (u, v) \ \in \ E(G_{1}) \quad \iff \quad (\mu(u), \mu(v)) \ \in \
	E(G_{2}) \text{,} \]
	which we will refer to as the \textit{map function}.
	
	\medskip
	
	Our approach will follow these steps:
	\begin{enumerate}
		\item Find a root \(r\) of \(T_{1}\).
		      
		      To do this, iterate over all vertices of \(T_{1}\), treating each
		      vertex as a candidate for \(r\). To verify a candidate, check for
		      each \(v \in V(T_{1}) \setminus \{r\}\) whether either \((v, r)
		      \in E(T_{1})\) or there exists a vertex \(u \in V(T_{1}) \setminus
		      \{v, r\}\) with \((v, u) \in E(T_{1})\) and \((u, r) \in
		      E(T_{1})\).
		      
		      This requires three binary counters (for \(r\), \(v\), and \(u\)).
		      If no root is found, we can reject the input.
		
		\item For each \(s \in V(T_{2})\), check if there exists a valid map
		      function \(\mu\) such that \(\mu(r) = s\).
		      
		      The remaining part of the solution will focus on this check.
	\end{enumerate}
	
	\underline{Lemma 1} Let \(T_{1}\) be a tree of radius \(2\) with root \(r\),
	and let \(T_{2}\) be any tree. Let \(v_{1}, \ldots, v_{k}\) denote the
	children of \(r\), ordered so that
	\[ \underset{i \in \{2, \ldots, k\}}{\forall} \ (\text{deg}(v_{i - 1}),
	v_{i - 1}) \ \geqslant \ (\text{deg}(v_{i}), v_{i}) \text{,} \]
	where
	\[ (a_{1}, b_{1}) \ \geqslant \ (a_{2}, b_{2}) \quad \iff \quad a_{1} >
	a_{2} \ \vee \ (a_{1} = a_{2} \ \wedge \ b_{1} \geqslant b_{2}) \text{.} \]
	Let \(s\) be any vertex of \(T_{2}\), and denote \(u_{1}, \ldots, u_{l}\)
	as the children of \(s\) when \(T_{2}\) is rooted at \(s\), ordered
	similarly. Then a map function \(\mu\) with the property that \(\mu(r) = s\)
	exists if and only if
	\begin{equation} \label{eq:eq1}
		k \ \leqslant \ l \quad \wedge \quad \underset{i \in \{1, \ldots, k\}}
		{\forall} \ \text{deg}(v_{i}) \ \leqslant \ \text{deg}(u_{i}) \text{.}
	\end{equation}
	
	\underline{Proof of Lemma 1} If condition \ref{eq:eq1} holds, we can define
	\(\mu(r) = s\) and set \(\mu(v_{i}) = u_{i}\) for each \(i \in \{1, \ldots,
	k\}\), ensuring there are enough vertices in \(T_{2}\) to accommodate each
	grandson of \(r\).
	
	\medskip
	
	To derive a contradiction, assume there is a map function \(\mu\), but
	condition \ref{eq:eq1} does not hold. Since \(k \leqslant l\) must be true,
	there exists an \(i \in \{1, \ldots, k\}\) such that \(\text{deg}(v_{i}) <
	\text{deg}(u_{i})\). Take the smallest such \(i\). There is a unique \(j\)
	such that \(\mu(v_{i}) = u_{j}\), which must be less than \(i\), or there
	would not be enough vertices to match the children of \(v_{i}\); thus \(i >
	1\). For each \(i' < i\), let \(j'\) be the index with \(\mu(v_{i'}) =
	u_{j'}\); similarly, \(j'\) must be less than \(i\) because \(\text{deg}(i')
	\geqslant \text{deg}(i)\). This creates a contradiction by the Pigeonhole
	principle, since \(\mu\) was assumed to be injective, completing the proof.
	
	\medskip
	
	Returning to the second step, once we have the root \(r\), finding the
	number \(k\) of its children requires only one binary counter. For a given
	\(s \in V(T_{2})\), we can similarly find the number of its children \(l\).
	By Lemma 1, if \(k > l\), we can discard \(s\) as a candidate. Otherwise, we
	need to verify for each \(i \in \{1, \ldots, k\}\) whether
	\(\text{deg}(v_{i}) \leqslant \text{deg}(u_{i})\). To calculate the degree
	of a vertex, we can iterate over edges while using a single binary counter.
	However, finding \(v_{i}\) and \(u_{i}\) in logarithmic space is more
	challenging.
	
	\medskip
	
	We can determine \((\text{deg}(v_{1}), v_{1})\) by iterating over the sons
	of \(r\), comparing each \((\text{deg}(v_{\text{cur}}), v_{\text{cur}})\)
	with the best pair found so far, and updating as needed. We can calculate
	\((\text{deg}(u_{1}), u_{1})\) in a similar way. After this, we compare
	\(\text{deg}(v_{1}) \leqslant \text{deg}(u_{1})\) and reject if the
	condition does not hold; otherwise, we move on to the next pair. Once
	\((\text{deg}(v_{i}), v_{i})\) is calculated, obtaining
	\((\text{deg}(v_{i + 1}), v_{i + 1})\) is straightforward -- simply ignore
	any \((\text{deg}(v_{\text{cur}}), v_{\text{cur}})\) strictly greater than
	\((\text{deg}(v_{i}), v_{i})\). The same process applies to
	\((\text{deg}(u_{i + 1}), u_{i + 1})\).
	
	\medskip
	
	The entire algorithm requires only a constant number of binary counters,
	concluding the solution.
\end{document}
