\documentclass[12pt]{article}
\usepackage[utf8]{inputenc}
\usepackage[T1]{fontenc}
\usepackage{indentfirst}
\usepackage{amsmath}
\usepackage{amssymb}
\usepackage{natbib}
\usepackage{graphicx}
\usepackage{float}
\usepackage[a4paper, margin = 2 cm]{geometry}
\usepackage{fancyhdr}
\usepackage{wrapfig}
\usepackage{hyperref}
\usepackage{mathtools}

\title{Computational complexity assignment 3}
\author{Dominik Wawszczak}
\date{2025-01-15}

\begin{document}
	\setlength{\parindent}{0 cm}
	
	Dominik Wawszczak \hfill Computational Complexity
	
	Student ID Number: 440014 \hfill Assignment 3
	
	Group Number: 3
	
	\bigskip
	\hrule
	\bigskip
	
	Let \(X, Y \subseteq \{0, 1\}^{k}\), where \(|X| = |Y| = n\), be an instance
	of the Orthogonal Vectors Problem. We construct two new sets of binary
	vectors with dimension \(d = 3k\):
	\[ A \ = \ \{xx0_{k} \ : \ x \in X\} \quad \text{and} \quad B \ = \
	\{\overline{y}0_{k}y \ : \ y \in Y\} \text{,} \]
	where \(vu\) denotes the concatenation of vectors \(v\) and \(u\), \(0_{l}\)
	is a vector consisting of \(l\) zeros, and \(\overline{v}\) is the vector
	\(v\) with every coordinate negated. Note that \(|A| = |B| = n\).
	
	\medskip
	
	We will prove the following equivalence:
	\begin{equation}
		\underset{x \in X}{\exists} \ \underset{y \in Y}{\exists} \ x \perp y
		\quad \iff \quad \underset{a \in A}{\exists} \
		\underset{b \in B}{\exists} \ \text{dist}_{\text{Hamming}}(a, b)
		\leqslant k \text{.} \label{eq:1}
	\end{equation}
	
	\medskip
	
	Assume there exist \(x \in X\) and \(y \in Y\) such that \(x \perp y\). Take
	\(a = xx0_{k} \in A\) and \(b = \overline{y}0_{k}y \in B\). Define
	\[ \text{cnt}_{p, q} \ = \ |\{i \ : \ x[i] = p \ \wedge \ y[i] = q\}|
	\text{,} \]
	for any \(p, q \in \{0, 1\}\). We have:
	\begin{align*}
		\text{dist}_{\text{Hamming}}(a, b) \ &= \
		\text{dist}_{\text{Hamming}}(xx0_{k}, \overline{y}0_{k}y) \ = \\
		&= \ \text{dist}_{\text{Hamming}}(x, \overline{y}) +
		\text{dist}_{\text{Hamming}}(x, 0_{k}) +
		\text{dist}_{\text{Hamming}}(0_{k}, y) \ = \\
		&= \ (\text{cnt}_{0, 0} + \text{cnt}_{1, 1}) + (\text{cnt}_{1, 0} +
		\text{cnt}_{1, 1}) + (\text{cnt}_{0, 1} + \text{cnt}_{1, 1}) \ = \\
		&= \ (k - \text{cnt}_{0, 1} - \text{cnt}_{1, 0}) + (\text{cnt}_{0, 1} +
		\text{cnt}_{1, 0} + 2 \cdot \text{cnt}_{1, 1}) \ = \ k + 2 \cdot
		\text{cnt}_{1, 1} \ = \ k \text{,}
	\end{align*}
	because \(\text{cnt}_{1, 1} = 0\), as \(x \perp y\). We also used the fact
	that \(\text{cnt}_{0, 0} + \text{cnt}_{0, 1} + \text{cnt}_{1, 0} +
	\text{cnt}_{1, 1} = k\).
	
	\medskip
	
	Conversely, assume there exist \(a \in A\) and \(b \in B\) such that
	\(\text{dist}_{\text{Hamming}}(a, b) \leqslant k\). Take \(x \in X\) and \(y
	\in Y\) where \(a = xx0_{k}\) and \(b = \overline{y}0_{k}y\). As before, we
	can show that \(\text{dist}_{\text{Hamming}}(a, b) = k + 2 \cdot
	\text{cnt}_{1, 1}\) with \(\text{cnt}_{p, q}\) defined as earlier.
	Therefore, it must hold that \(\text{cnt}_{1, 1} = 0\), which means \(x
	\perp y\), completing the proof of the equivalence \eqref{eq:1}.
	
	\medskip
	
	Suppose there exists a constant \(\delta > 0\) such the problem from the
	statement can be solved in time \(\mathcal{O} \big( d^{100} \cdot n^{2 -
	\delta} \big)\). Then, due to the equivalence \eqref{eq:1}, the Orthogonal
	Vectors Problem could be solved in time
	\[ \mathcal{O} \big( d^{100} \cdot n^{2 - \delta} \big) \ = \ \mathcal{O}
	\big( (3k)^{100} \cdot n^{2 - \delta} \big) \ = \ k^{\mathcal{O}(1)} \cdot
	n^{2 - \delta} \text{,} \]
	which contradicts the Orthogonal Vectors Conjecture, concluding the proof,
	since
	\[ \text{Strong Exponential Time Hypothesis} \ \implies \
	\text{Orthogonal Vectors Conjecture.} \]
\end{document}
