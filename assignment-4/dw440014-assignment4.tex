\documentclass[12pt]{article}
\usepackage[utf8]{inputenc}
\usepackage[T1]{fontenc}
\usepackage{indentfirst}
\usepackage{amsmath}
\usepackage{amssymb}
\usepackage{natbib}
\usepackage{graphicx}
\usepackage{float}
\usepackage[a4paper, margin = 2 cm]{geometry}
\usepackage{fancyhdr}
\usepackage{wrapfig}
\usepackage{hyperref}
\usepackage{mathtools}

\title{Computational complexity assignment 4}
\author{Dominik Wawszczak}
\date{2025-01-17}

\begin{document}
	\setlength{\parindent}{0 cm}
	
	Dominik Wawszczak \hfill Computational Complexity
	
	Student ID Number: 440014 \hfill Assignment 4
	
	Group Number: 3
	
	\bigskip
	\hrule
	\bigskip
	
	Consider an instance \(G^{\ast}\) of the \(3\)-Coloring Problem, with
	\(V(G^{\ast}) = N\). Set \(k = \lfloor \log_{3} \log_{2} N \rfloor -
	1\)\footnote{We assume that \(N \geqslant 512\), as smaller values would
	make \(k\) either non-positive or undefined.}. We partition the vertices of
	\(G^{\ast}\) into groups of size \(k\), with the final group potentially
	containing fewer than \(k\) vertices.
	
	\medskip
	
	We construct a graph \(G\) -- an instance of the problem from the statement.
	Let \(n\) denote the number of groups. We define \(V(G)\) as
	\(\{1, \ldots, n\} \times \{1, \ldots, n\}\). It holds that
	\[ n \ \leqslant \ \frac{N}{k} + 1 \ = \
	\frac{N}{\lfloor \log_{3} \log_{2} N \rfloor - 1} + 1 \ \leqslant \
	\frac{\alpha N}{\log_{3} \log_{2} N} \text{,} \]
	for some constant \(\alpha\), because
	\[ \frac{N}{\lfloor \log_{3} \log_{2} N \rfloor - 1} + 1 \ \leqslant \
	\frac{\alpha N}{\log_{3} \log_{2} N} \quad \iff \quad
	\frac{\log_{3} \log_{2} N}{\lfloor \log_{3} \log_{2} N \rfloor - 1} +
	\frac{\log_{3} \log_{2} N}{N} \ \leqslant \ \alpha \text{,} \]
	which is true since \(\frac{x}{\lfloor x \rfloor - 1} < 3\) for \(x
	\geqslant 2\), and \(\frac{\log_{3} \log_{2} N}{N}\) is bounded by a
	constant, as it converges to \(0\). Thus, \(n = o(N)\).
	
	\medskip
	
	Let \(l_{i}\) denote the number of possible \(3\)-colorings of the \(i\)-th
	group (\(1 \leqslant i \leqslant n\)). It follows that
	\[ l_{i} \ \leqslant \ 3^{k} \ = \
	3^{\lfloor \log_{3} \log_{2} N \rfloor - 1} \ \leqslant \
	3^{\log_{3} \log_{2} N - 1} \ = \ \frac{\log_{2} N}{3} \ = \
	\log_{2} \sqrt[3]{N} \ < \ \log_{2} n \text{,} \]
	because
	\[ n \ \geqslant \ \frac{N}{k} \ = \
	\frac{N}{\lfloor \log_{3} \log_{2} N \rfloor - 1} \ > \ \sqrt[3]{N}
	\text{.} \]
	
	\medskip
	
	Consider an arbitrary ordering of all the colorings in each group, such as
	lexicographic order. For every \(1 \leqslant i_{1}, i_{2} \leqslant n\) with
	\(i_{1} \neq i_{2}\), and for every \(1 \leqslant j_{1} \leqslant l_{i_{1}},
	\ 1 \leqslant j_{2} \leqslant l_{i_{2}}\), we create an edge between
	vertices \((i_{1}, j_{1})\) and \((i_{2}, j_{2})\) if and only if the
	colorings \(j_{1}\) and \(j_{2}\) are compatible. This construction
	satisfies both conditions stated in the problem.
	
	\medskip
	
	We will prove the following equivalence:
	\begin{equation}
		G^{\ast} \ \text{is \(3\)-colorable} \quad \iff \quad G \
		\text{contains a clique of size} \ n \text{.} \label{eq:1}
	\end{equation}
	Assume \(G^{\ast}\) is \(3\)-colorable. Then, the vertices representing the
	partial colorings of each group form a clique. Conversely, if \(G\) contains
	a clique of size \(n\), a valid \(3\)-coloring of \(G^{\ast}\) can be
	constructed by selecting the corresponding colorings for every group.
	
	\medskip
	
	Suppose there exists a constant \(c\) such that the problem from the
	statement can be solved in time \(\mathcal{O}(c^{n}) =
	\mathcal{O} \big( 2^{\lambda n} \big)\), where \(\lambda = \log_{2} c\).
	This implies that the \(3\)-Coloring Problem could be solved in time
	\(\mathcal{O} \big( 2^{\lambda n} \big) = \mathcal{O}\big( 2^{o(N)} \big)\),
	due to the equivalence \eqref{eq:1}. However, by the Sparsification Lemma
	and the linear reduction from \(3\)-SAT to \(3\)-Coloring, this contradicts
	the Exponential Time Hypothesis, completing the proof.
\end{document}
